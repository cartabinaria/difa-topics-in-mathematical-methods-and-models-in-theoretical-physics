\section{Teoria dei Gruppi}
\subsection{Definizioni}

Diamo la definizione di \textit{gruppo}:

un gruppo $G$ è un insieme di elementi $\{g_i\}$ dotato di un legge di composizione interna $\cdot$ che gode delle seguenti proprietà:
\begin{enumerate}
    \item chiusura: $\forall g_i,g_j\in G \quad:\quad g_i\cdot g_j\in G$;
    \item associatività: $g_i\cdot (g_j\cdot g_k)=g_k\cdot (g_i\cdot g_j)$;
    \item identità: $\forall g_i\in G, \quad\exists g_0\in G \quad : \quad  g_i\cdot g_0= g_0\cdot g_i=g_i $;
    \item inverso: $\forall g_i\in G, \quad\exists g_i^{-1}\in G \quad : \quad  g_i\cdot g_i^{-1}= g_i^{-1}\cdot g_i=g_0$.
\end{enumerate}
inoltre, se vale la proprietà commutativa $g_i\cdot g_j=g_j\cdot g_i$ il gruppo $G$ è detto \textit{abeliano}.

Definiamo una \textit{rappresentazione del gruppo} come un insieme di di operatori lineari $\{T_i\}$, il quale è un omomorfismo che associa ad ogni elemento del gruppo un operatore lineare.

Quindi, $\forall g_i\in G\xrightarrow{omo}T(g_i):L\xrightarrow{}L$, ove $L$ è uno spazio lineare. Gli operatori riproducono le proprietà del gruppo di partenza:
\begin{enumerate}
    \item  $T(g_i)T(g_j)=T(g_i\cdot g_j)\qquad\forall i,j$;
    \item $T(g_0)=\mathds{I}$;
    \item  $T(g_i^{-1})=T^{-1}(g_i) \qquad\forall i$.
\end{enumerate}
Inoltre, diremo che la dimensione della rappresentazione è pari alla dimensione dello spazio vettoriale. Le rappresentazioni si caratterizzano per quelle finite dimensionali e per quelle infinite dimensionali.

\subsection{Gruppo per rotazioni}
I primi gruppi di interesse in fisica sono quelli che rappresentano rotazioni. In particolare, in forma matriciale e vettoriale abbiamo:
\begin{equation}
\begin{pmatrix}
x'   \\
y'   \\
 z'  \\
\end{pmatrix}
=R\begin{pmatrix}
x \\
y   \\
 z  \\
\end{pmatrix}\qquad
\Vec{r'}=R\Vec{r}
\end{equation}
ove $R$ è una matrice $3\times3$. Sotto rotazioni sappiamo che deve preservare l'invariante:
\begin{equation}
\begin{gathered}
    x'^2+y'^2+z'^2=x^2+y^2+z^2 \implies \Vec{r'}^T\Vec{r'}=\Vec{r}^T\Vec{r}\implies\\
    \Vec{r}^TR^TR\Vec{r}=\Vec{r}^T\Vec{r}\implies R^TR=\mathds{I}\implies R^T=R^{-1}
\end{gathered}
\end{equation}
Matrici $3\times3$ di questo tipo sono dette \textit{ortogonali}. Tutte le proprietà della definizione di gruppo sono soddisfatte, quindi definiamo il gruppo $O(3)$ di matrici ortogonali $\{R\}$.

Considerando il determinante della relazione matriciale precedente:
\begin{equation}   \det\left[R^TR\right]=1\implies \det\left[R\right]^2=1 \implies \det\left[R\right]=\pm1
\end{equation}
se il determinate è $-1$ si ha l'inversione spaziale, ossia una trasformazione discreta, che è la parità. Mentre per determinate pari a $1$ si hanno le trasformazioni dette speciali e il gruppo associato è $SO(3)$.
Noi ci limiteremo a quest'ultimo gruppo.

Consideriamo l'esempio di una rotazione nel piano $x-y$: 
\begin{equation}
    \Vec{v}'=R_z(\theta)\Vec{v}\qquad
    \begin{pmatrix}
v_x'   \\
v_y'   \\
 v_z'  \\
\end{pmatrix}
=    \begin{pmatrix}
 \cos{\theta}&\sin{\theta}&0  \\
  -\sin{\theta}&\cos{\theta}&0\\
0&0&1\\
\end{pmatrix}
\begin{pmatrix}
v_x \\
v_y   \\
 v_z  \\
\end{pmatrix}
\end{equation}
In maniera analoga possiamo definire le rotazioni attorno agli assi $x$ e $y$, le cui matrici associate sono:
\begin{equation}
  R_x(\phi)= \begin{pmatrix}
  1&0&0\\
0& \cos{\phi}&\sin{\phi} \\
 0& -\sin{\phi}&\cos{\phi}\\
\end{pmatrix}\qquad
  R_y(\psi)= \begin{pmatrix}
 \cos{\psi}&0&0-\sin{\psi} \\
 0&1&0\\
  \sin{\psi}&0&\cos{\psi}\\
\end{pmatrix}
\end{equation}
si verifica immediatamente che $ R_x(\phi)\cdot R_y(\psi)\neq R_y(\psi)\cdot R_x(\phi)$, quindi il gruppo di rotazioni non è abeliano.

Sappiamo che una rotazione generica può sempre essere scomposta in tre rotazioni elementari attorno agli assi, da cui segue che, in generale, una rotazione dipende da tre parametri\footnote{Ciò non è nuovo, un esempio sono gli angoli di Eulero.}. Quindi, ogni elemento del gruppo dipende da tre parametri, e per questo il gruppo viene detto \textit{a tre parametri}\footnote{Notiamo il fatto che il numero di parametri sia uguale alla dimensione della rappresentazione non è una legge, si tratta di un caso. Infatti, più avanti, vedremo esempi nei quali ciò non avviene.}. 
Poiché le rotazioni sono funzioni continue, avremo che il gruppo di rotazioni ha infiniti elementi. Inoltre abbiamo che gli elementi del gruppo $SO(3)$ sono funzioni classe $C^\infty$, un tale gruppo è un \textit{gruppo di Lie}.

In generale, le componenti indipendenti di una matrice $3\times3$ sono $9$ , tuttavia abbiamo imposto la condizione $R^TR=\mathds{I}$ la quale coinvolge $6$ relazioni e quindi abbiamo $9-6=3$, che è la dimensione della rappresentazione.

Introduciamo il concetto di \textit{generatore del gruppo}, ossia una rotazione infinitesima. In particolare i generatori del gruppo $SO(3)$\footnote{Non consideriamo l'elemento del gruppo $O(3)$ che ha determinate pari a $-1$, poiché nella definizione di generatore è presente la variazione continua dei parametri. Essendo la parità discreta ciò non è possibile.} sono:
\begin{equation}
    J_z=\dfrac{1}{i}\dfrac{dR_z}{d\theta}(\theta)\biggm|_{\theta=0}= 
    \begin{pmatrix}
         0&-i&0  \\
  i&0&0\\
0&0&0\\
    \end{pmatrix}
\end{equation}
\begin{equation}
    J_x=\dfrac{1}{i}\dfrac{dR_x}{d\phi}(\phi)\biggm|_{\phi=0}= 
    \begin{pmatrix}
         0&0&0  \\
  0&0&-i\\
0&i&0\\
    \end{pmatrix}
\end{equation}
\begin{equation}
    J_y=\dfrac{1}{i}\dfrac{dR_y}{d\psi}(\psi)\biggm|_{\psi=0}= 
    \begin{pmatrix}
         0&0&i  \\
  0&0&0\\
-i&0&0\\
    \end{pmatrix}
\end{equation}
notiamo che tali matrici sono tutte hermitiane e non commutano. Considerando il commutatore abbiamo:
\begin{equation}\phantomsection\label{eq:alg_lie_so3}
    \left[J_i,J_j\right]=i\epsilon_{ijk}J_k
\end{equation}
ove $\epsilon_{ijk}$ sono dette \textit{costanti di struttura}\footnote{Le costanti di struttura di un gruppo sono indipendenti dalla rappresentazione.}. La relazione \eqref{eq:alg_lie_so3} è detta \textit{algebra di Lie degli operatori di $SO(3)$}.

Abbiamo appena visto che, in generale, i generatori di un gruppo non commutano, possiamo, però, introdurre l'\textit{operatore di Casimir} $J^2=J_x^2+J_y^2+J_z^2$ il quale, per definizione, commuta con tutti i generatori del gruppo.
\begin{equation}
    \left[J^2,J_i\right]=0 \qquad \forall i
\end{equation}

Ora mostriamo come sia possibile, a partire dai generatori, costruire gli elementi del gruppo. Dalla definizione di generatore e considerando la rotazione infinitesima $\delta \theta$, abbiamo:
\begin{equation}
    R_z(\delta\theta)=\mathds{I}+iJ_z\delta\theta
\end{equation}
considerando un angolo finito $\theta=N\delta\theta$ per $N\xrightarrow{}\infty$, sappiamo che è possibile esprimere una rotazione di un angolo $\theta$ come una composizione di  $N$ rotazioni infinitesime\footnote{In particolare abbiamo che la rotazione finita è esprimibile come prodotto delle rotazioni infinitesime.}:
\begin{equation}
    R_z(\theta)=\left[R_z(\delta\theta)\right]^N=\left[\mathds{I}+iJ_z\delta\theta\right]^N=\left[\mathds{I}+iJ_z\dfrac{\theta}{N}\right]^N=e^{iJ_z\theta}
\end{equation}
sviluppando l'esponenziale in serie:
\begin{equation}
    e^{iJ_z\theta}=\mathds{I}+iJ_z\theta-J_z^2\dfrac{\theta^2}{2!}-iJ_z^3\dfrac{\theta^3}{3!}+\cdots
\end{equation}
in formalismo matriciale:

\begin{equation*}
    \begin{pmatrix}
 \cos{\theta}&\sin{\theta}&0  \\
  -\sin{\theta}&\cos{\theta}&0\\
0&0&1\\
\end{pmatrix}= \begin{pmatrix}
 1&0&0  \\
 0&1&0\\
0&0&1\\
\end{pmatrix}+\theta\begin{pmatrix}
 0&1&0  \\
 -1&0&0\\
0&0&0\\
\end{pmatrix}+\dfrac{\theta^2}{2!}\begin{pmatrix}
 -1&0&0  \\
0 &-1&0\\
0&0&0\\
\end{pmatrix}
\end{equation*}
Possiamo generalizzare le relazioni ottenute considerando la rotazione nel piano ortogonale al vettore $\hat{n}$:
\begin{equation}
    \Vec{\theta}=\hat{n}\theta; \qquad R_n(\theta)=e^{i\Vec{J}\Vec{\theta}}
\end{equation}


Consideriamo, ora, un nuovo gruppo, il quale, dimostreremo, rappresenta anch'esso le rotazioni.
Il gruppo è $SU(2)$, quindi di matrici $2\times2$, speciali e unitarie\footnote{Ricordiamo che la definizione di unitarietà, presuppone la proprietà $UU^+=\mathds{I}$, quindi $U^+=U^{-1}$.}. Una generica matrice avrà forma:
\begin{equation}\phantomsection\label{eq:mat_uni}
  U=\begin{pmatrix}
 a&b \\
c &d\\
\end{pmatrix}
\end{equation}
ove gli elementi della matrice sono, in generale, complessi. Applicando l'aggiunzione\footnote{Ricordiamo che l'aggiunzione consiste nella trasposizione coniugata.}, otteniamo:
\begin{equation}\phantomsection\label{eq:mat_uni_agg}
  U^+=\begin{pmatrix}
 a^*&c^* \\
b^* &d^*\\
\end{pmatrix}
\end{equation}
Sappiamo che la matrice inversa della \eqref{eq:mat_uni} è
\begin{equation}\phantomsection\label{eq:mat_uni_inv}
  U^{-1}=\begin{pmatrix}
 d&-b \\
-c &a\\
\end{pmatrix}
\end{equation}
poiché sappiamo che la matrice aggiunta e quella inversa devono essere uguali per la proprietà di unitarietà, concludiamo che la matrice dovrà avere una forma del tipo:
\begin{equation}
  U=\begin{pmatrix}
 a&b \\
-b^* &a^*\\
\end{pmatrix}
\end{equation}
quindi siamo passati da 4 numeri complessi a 2, quindi 4 numeri reali. Dalla condizione si specialità abbiamo che il determinate sarà:
\begin{equation}
    |a|^2+|b|^2=1
\end{equation}
quest'ultima relazione mostra che i numeri complessi $a$ e $b$ non sono indipendenti, in particolare avremo che i parametri reali indipendenti saranno $4-1=3$, i quali sono i parametri necessari alla descrizione delle rotazioni.

Definiamo uno \textit{spinore} come un vettore bidimensionale complesso del tipo:
\begin{equation}
    \xi=\begin{pmatrix}
 \xi_1 \\
\xi_2\\
\end{pmatrix}
\end{equation}
possiamo definire la trasformazione $\xi\xrightarrow{}\xi'=U\xi$:
\begin{equation}
   \begin{pmatrix}
 \xi_1' \\
\xi_2'\\
\end{pmatrix}=\begin{pmatrix}
 a&b \\
-b^* &a^*\\
\end{pmatrix} \begin{pmatrix}
 \xi_1 \\
\xi_2\\
\end{pmatrix}
\end{equation}
abbiamo che le componenti saranno:
\begin{equation}
    \begin{cases}
        \xi_1'=a\xi_1+b\xi_2\\
        \xi_2'=-b^*\xi_1+a^*\xi_2
    \end{cases}
    \end{equation}
Consideriamo il vettore tridimensionale:
\begin{equation}
    \Vec{r}=\begin{pmatrix}
 x \\
y\\
z
\end{pmatrix}
\end{equation}
a partire da questo possiamo costruire l'oggetto:
\begin{equation}
   h=\Vec{\sigma}\cdot \Vec{r}=\begin{pmatrix}
 z &x-iy\\
x+iy&-z\\
\end{pmatrix}
\end{equation}
ove $\Vec{\sigma}$ rappresenta un vettore che ha per componenti le matrici di Pauli:
\begin{equation}
  \sigma_x=\begin{pmatrix}
 0 &1\\
1&0\\
\end{pmatrix}; \sigma_y=\begin{pmatrix}
 0&-i\\
i&0\\
\end{pmatrix}; \sigma_z=\begin{pmatrix}
 1&0\\
0&-1\\
\end{pmatrix}
\end{equation}
Notiamo che $h$ è una matrice hermitaina con traccia nulla con determinante $\det (h)=x^2+y^2+z^2$.
A questo punto, definiamo la trasformazione $h\xrightarrow{}h'=UhU^+$, ove $U\in SU(2)$. Si può dimostrare che anche $h'$ è hermitiana e a traccia nulla\footnote{Come prima, abbiamo una matrice matrice con 4 elemnti complessi, quindi 8 reali. Le condizioni di hermiticità e di traccia nulla implicano che saranno solo 3 ad essere indipendenti.}.

Possiamo quindi scrivere:
\begin{equation}\phantomsection\label{eq:h_trasf}
   h'=\begin{pmatrix}
 z' &x'-iy'\\
x'+iy'&-z'\\
\end{pmatrix}=
\begin{pmatrix}
 a&b \\
-b^* &a^*\\
\end{pmatrix} 
\begin{pmatrix}
 z &x-iy\\
x+iy&-z\\
\end{pmatrix}
\begin{pmatrix}
 a^*&-b \\
b^* &a\\
\end{pmatrix} 
\end{equation}
e il suo determinante $\det(h')=\det(UhU^+)=\det(U)\det(h)\det(U^+)=\det(h)$
Quindi la trasformazione definita sopra mantiene invariata la forma quadratica $x^2+y^2+z^2$:
\begin{equation}
    x^2+y^2+z^2=x'^2+y'^2+z'^2
\end{equation}
tale proprietà è congruente con la proprietà delle rotazioni, quindi questo tipo di trasformazioni \say{simula} le rotazioni in $\mathds{R}^3$.

Sviluppando i calcoli della \eqref{eq:h_trasf} otteniamo le componenti:
\begin{equation}
    \begin{cases}
        x'=\dfrac{1}{2}(a^2+a^{*2}-b^2-b^{*2})x-\dfrac{i}{2}(a^2-a^{*2}+b^2-b^{*2})y-(a^*b^*+ab)z\\
        y'=\dfrac{i}{2}(a^2-a^{*2}-b^2+b^{*2})x+\dfrac{1}{2}(a^2+a^{*2}+b^2+b^{*2})y-i(ab-a^*b^*)z\\
        z'=(ab^*+b^*a)x+i(ba^*-ab^*)y+(|a|^2-|b|^2)z
    \end{cases}
\end{equation}

Ora, consideriamo tre esempi, ricordiamo che $a$ e $b$ devono soddisfare $|a|^2+|b|^2=1$.
\begin{enumerate}
    \item  Consideriamo $a=e^{i\frac{\alpha}{2}}$ e $b=0$ da cui otteniamo:
    \begin{equation}
        \begin{cases}
             x'=x\cos{\alpha}+y\sin{\alpha}\\
        y'=-x\sin{\alpha}+y\cos{\alpha}\\
        z'=z
        \end{cases}
    \end{equation}
quindi le matrici che rappresentano la stessa rotazione sono:
\begin{equation}
        U=\begin{pmatrix}
e^{i\frac{\alpha}{2}}  &0\\
0&e^{-i\frac{\alpha}{2}}\\
\end{pmatrix} \qquad\Longleftrightarrow\qquad R_z(\alpha)= \begin{pmatrix}
 \cos{\alpha}&\sin{\alpha}&0  \\
  -\sin{\alpha}&\cos{\alpha}&0\\
0&0&1\\
\end{pmatrix}
    \end{equation}
    considerando i generatori abbiamo:
    \begin{equation}
        U=e^{i\sigma_z\frac{\alpha}{2}} \qquad\Longleftrightarrow\qquad R_z(\alpha)= e^{iJ_z\alpha}
    \end{equation}



    
    \item Consideriamo $a=\cos{\frac{\beta}{2}}$ e $b=\sin{\frac{\beta}{2}}$ da cui otteniamo le matrici che rappresentano la rotazione nel piano x-z sono:
\begin{equation}
        U=\begin{pmatrix}
\cos{\frac{\beta}{2}}  &\sin{\frac{\beta}{2}}\\
-\sin{\frac{\beta}{2}}&\cos{\frac{\beta}{2}}\\
\end{pmatrix} \qquad\Longleftrightarrow\qquad  R_y(\beta)= \begin{pmatrix}
 \cos{\beta}&0&0-\sin{\beta} \\
 0&1&0\\
  \sin{\beta}&0&\cos{\beta}\\
\end{pmatrix}
    \end{equation}
    considerando i generatori abbiamo:
    \begin{equation}
        U=e^{i\sigma_y\frac{\beta}{2}} \qquad\Longleftrightarrow\qquad R_y(\beta)= e^{iJ_y\beta}
    \end{equation}



    
    \item Consideriamo $a=\cos{\frac{\gamma}{2}}$ e $b=i\sin{\frac{\gamma}{2}}$ da cui otteniamo le matrici che rappresentano la rotazione nel piano y-z sono:
\begin{equation}
        U=\begin{pmatrix}
\cos{\frac{\gamma}{2}}  &i\sin{\frac{\gamma}{2}}\\
-i\sin{\frac{\gamma}{2}}&\cos{\frac{\gamma}{2}}\\
\end{pmatrix} \qquad\Longleftrightarrow\qquad   R_x(\gamma)= \begin{pmatrix}
  1&0&0\\
0& \cos{\gamma}&\sin{\gamma} \\
 0& -\sin{\gamma}&\cos{\gamma}\\
\end{pmatrix}
    \end{equation}
    considerando i generatori abbiamo:
    \begin{equation}
        U=e^{i\sigma_x\frac{\gamma}{2}} \qquad\Longleftrightarrow\qquad R_x(\gamma)= e^{iJ_x\gamma}
    \end{equation}
\end{enumerate}
Concludiamo che in generale:
\begin{equation}
        U=e^{i\Vec{\sigma}\frac{\Vec{\theta}}{2}} \qquad\Longleftrightarrow\qquad R_n= e^{i\Vec{J}\Vec{\theta}}
    \end{equation}
osservando quest'ultima relazione vediamo che i generatori $\Vec{J}$ di $SO(3)$ hanno come controparte, nella rappresentazione di $SU(2)$, $\dfrac{\Vec{\sigma}}{2}$. Capiamo, allora, che questi sono i generatori di tale gruppo e come tali possiamo definire un'algebra di Lie\footnote{La quale sarà la medesima della rappresentazione $SO(3).$}:
\begin{equation}\phantomsection\label{eq:alg_lie_su2}
    \left[\dfrac{\sigma_i}{2},\dfrac{\sigma_j}{2}\right]=i\epsilon_{ijk}\dfrac{\sigma_k}{2}
\end{equation}

C'è un'ultima importante osservazione da fare.
Le due rappresentazioni qui studiate non sono uguali o, meglio, non vi è, tra le due, una corrispondenza biunivoca.
Se consideriamo una rotazione di $2\pi$: $\theta\xrightarrow{}\theta+2\pi$, avremo che la rappresentazione $SO(3)$ associa la medesima matrice prima e dopo la rotazione: $R\xrightarrow{}R$. Al contrario la rappresentazione $SU(2)$ fornisce la matrice opposta $U\xrightarrow{}-U$. Pertanto a una matrice $R\in SO(3)$ sono associabili due matrici $-U,U\in SU(2)$; tutte e tre queste matrici rappresentano la medesima rotazione nello spazio tridimensionale.

\subsection{Gruppo di Lorentz}
Consideriamo le trasformazioni di Lorentz, ampiamente studiate nella sezione \ref{sec:1.1}, le quali danno origine al gruppo di matrici ortogonali $O(3,1)$.
\begin{equation}
  x^{\alpha}\xrightarrow[\text{}]{\text{T}}x'^{\beta}=\Lambda\indices{^\beta_\nu} x^{\nu}
\end{equation}
In forma matriciale, sappiamo che le matrici associate a queste trasformazioni rispettano la proprietà: 
\begin{equation}
\Lambda^T\eta\Lambda=\eta
\end{equation}
ove, ricordiamo, $\eta$ è la matrice metrica. Considerando i determinati, otteniamo:
\begin{equation}
\det(\Lambda^T\eta\Lambda)=1 \implies \det\Lambda=\pm1
\end{equation}
Inoltre, si distinguono per quelle ortocrone con $\Lambda^0_0\geq 1$ e quelle non ortocrone $\Lambda^0_0\leq -1$. Sappiamo che tali trasformazioni mantengono invariata la quantità:
\begin{equation}
    c^2t^2-x^2-y^2-z^2=  c^2t'^2-x'^2-y'^2-z'^2
\end{equation}
Noi ci occuperemo in particolare del gruppo delle matrici con determinate $1$ ed ortocrone, che viene detto $SO(3,1)$.
Le trasformazioni con determinate pari a $-1$ sono, per esempio, quella di \textit{inversione temporale} $(t\xrightarrow{}-t;\Vec{x}\xrightarrow{}\Vec{x})$ o quella di \textit{parità} $(t\xrightarrow{}t;\Vec{x}\xrightarrow{}-\Vec{x})$:
\begin{equation}
T= \begin{pmatrix}
  -1&0&0&0\\
0& 1&0&0 \\
0&0&1&0 \\
 0& 0&0&1\\
\end{pmatrix} \qquad P= \begin{pmatrix}
  1&0&0&0\\
0& -1&0&0 \\
0&0&-1&0 \\
 0& 0&0&-1\\
\end{pmatrix}
    \end{equation}

Il gruppo delle rotazioni studiate nella sezione precedenti risulta essere un sottogruppo del gruppo di Lorentz $SO(3,1)$, in particolare avremo che:
\begin{equation}
    \Lambda=\begin{pmatrix}
     1&0&0&0\\
     0&&&\\
      0&&R&\\
    0&&&\\
\end{pmatrix}
\end{equation}

Consideriamo, ora, un boost lungo $x$, per cui avremo che la trasformazione è rappresentata dalla matrice:
\begin{equation}
\begin{pmatrix}
    x'^0 \\
    x'^1 \\
    x'^2 \\
    x'^4 \\
\end{pmatrix}=
\begin{pmatrix}
\gamma & -\gamma\beta & 0 & 0   \\
 -\gamma\beta &\gamma & 0 & 0    \\
  0 & 0 & 1 & 0                   \\
  0 & 0 & 0 & 1
\end{pmatrix}\begin{pmatrix}
    x^0 \\
    x^1 \\
    x^2 \\
    x^4 \\
\end{pmatrix}
\end{equation}
sappiamo che deve valere la relazione $\gamma^2-\beta^2\gamma^2=1$, quindi possiamo parametrizzare $\gamma=\cosh{\phi}$ e $\gamma \beta=\sinh{\phi}$
\begin{equation}
\begin{pmatrix}
    x'^0 \\
    x'^1 \\
    x'^2 \\
    x'^4 \\
\end{pmatrix}=
\begin{pmatrix}
\cosh{\phi} & \sinh{\phi} & 0 & 0   \\
 \sinh{\phi} &\cosh{\phi} & 0 & 0    \\
  0 & 0 & 1 & 0                   \\
  0 & 0 & 0 & 1
\end{pmatrix}\begin{pmatrix}
    x^0 \\
    x^1 \\
    x^2 \\
    x^4 \\
\end{pmatrix}
\end{equation}
Similmente a quanto fatto precedentemente, possiamo definire i generatori dei boost rispetto alle tre direzioni:
\begin{equation}
\begin{gathered}
    K_x=\dfrac{1}{i}\dfrac{dB_x}{d\phi}(\phi)\biggm|_{\phi=0}= 
   -i \begin{pmatrix}
         0&1&0&0  \\
  1&0&0&0\\
0&0&0&0\\
0&0&0&0\\
    \end{pmatrix}\\
      K_y=\dfrac{1}{i}\dfrac{dB_y}{d\psi}(\psi)\biggm|_{\psi=0}= 
   -i \begin{pmatrix}
         0&0&1&0  \\
  0&0&0&0\\
1&0&0&0\\
0&0&0&0\\
    \end{pmatrix}\\
      K_z=\dfrac{1}{i}\dfrac{dB_z}{d\theta}(\theta)\biggm|_{\theta=0}= 
   -i \begin{pmatrix}
         0&0&0&1 \\
  0&0&0&0\\
0&0&0&0\\
1&0&0&0\\
    \end{pmatrix}
\end{gathered}
\end{equation}
osserviamo che queste non sono hermitiane. Mentre i generatori per le rotazioni nel formalismo quadridimensionale sono:
\begin{equation}
\begin{gathered}
    J_x= -i
    \begin{pmatrix}
         0&0&0&0\\
  0&0&0&0\\
0&0&0&1\\
0&0&-1&0\\
    \end{pmatrix}
\\
    J_y= -i
   \begin{pmatrix}
         0&0&0&0\\
  0&0&0&-1\\
0&0&0&0\\
0&1&0&0\\
    \end{pmatrix}\\
     J_z=-i 
    \begin{pmatrix}
         0&0&0&0\\
  0&0&1&0\\
0&-1&0&0\\
0&0&0&0\\
    \end{pmatrix}
    \end{gathered}
\end{equation}
Quindi una generica matrice $\Lambda$ sarà definita a partire da 6 generatori.
Inoltre, l'algebra di Lie dei generatori:

\begin{equation}
    \left[J_i,J_j\right]=i\epsilon_{ijk}J_k
\end{equation}

\begin{equation}
    \left[J_i,K_j\right]=i\epsilon_{ijk}K_k
\end{equation}

\begin{equation}
    \left[K_i,K_j\right]=-i\epsilon_{ijk}J_k
\end{equation}
si suole dire che l'algebra dei $K_i$ non è chiusa, mentre quella dei $J_i$ è essere chiusa; questo a causa dei risultati dei commutatori.

Possiamo introdurre due diversi operatori di Casimir:
\begin{equation}
   |\Vec{J}|^2-|\Vec{K}|^2 \quad \text{e}\quad \Vec{J}\Vec{K}=J_iK_i
\end{equation}

La rappresentazione studiata fino ad ora è una rappresentazione quadrimensionale, ora introduciamo una rappresentazione infinito dimensionale. In particolare, la prima che analizzeremo è per il gruppo delle rotazioni.

Consideriamo una rotazione nel piano $x-y$:
\begin{equation}
    \begin{cases}
        x'=x\cos{\theta}+y\sin{\theta} \\
        y'=-x\sin{\theta}+y\cos{\theta} \\
        z'=z
    \end{cases}
\end{equation}
per una rotazione infinitesima, quindi per $\theta\xrightarrow{}0$, abbiamo:
\begin{equation}
    \begin{cases}
        x'=x+y\theta \\
        y'=-x\theta+y \\
        z'=z
    \end{cases}
\end{equation}
i differenziali sono:
\begin{equation}
    \begin{gathered}
        dx=x'-x=y\theta\\
        dy=y'-y=-x\theta
    \end{gathered}
\end{equation}

Consideriamo, ora, una generica funzione $f(x,y,z)$ a cui applichiamo il generatore di una rotazione attorno all'asse $z$, avremo:
\begin{equation}
    \begin{aligned}
        J_zf(x,y,z)=&-i\lim_{\theta\xrightarrow{}0} \dfrac{1}{\theta}[f(x',y',z')-f(x,y,z)]\\
        &=-i\lim_{\theta\xrightarrow{}0} \dfrac{1}{\theta}[f(x+y\theta,-x\theta+y,z)-f(x,y,z)]\\
        &=-i\lim_{\theta\xrightarrow{}0} \dfrac{1}{\theta}\left(\dfrac{\partial f}{\partial x}dx+\dfrac{\partial f}{\partial y}dy\right)=-i\lim_{\theta\xrightarrow{}0} \dfrac{1}{\theta}\left[\dfrac{\partial f}{\partial x}(y\theta)+\dfrac{\partial f}{\partial y}(-x\theta)\right]\\
        &=-i\left(\dfrac{\partial f}{\partial x}y-\dfrac{\partial f}{\partial y}x\right)
    \end{aligned}
\end{equation}
elidendo le $f$ e generalizzando a tutti i generatori\footnote{Notiamo che i generatori hanno la medesima forma degli operatori momento angolare in meccanica quantistica, quindi l'operatore momento angolare è il generatore delle rotazioni.}:
\begin{equation}
    \begin{gathered}
        J_z=-i\left(y\dfrac{\partial }{\partial x}-x\dfrac{\partial }{\partial y}\right)\\
          J_x=-i\left(y\dfrac{\partial }{\partial z}-z\dfrac{\partial }{\partial y}\right)\\
            J_y=-i\left(z\dfrac{\partial }{\partial x}-x\dfrac{\partial }{\partial z}\right)
    \end{gathered}
\end{equation}
per i quali vale l'algebra di Lie: 
\begin{equation}
    \left[J_i,J_j\right]=i\epsilon_{ijk}J_k
\end{equation}
Possiamo generalizzare il concetto di generatore, considerando una parametrizzazione $a^i$ generica:
\begin{equation}\phantomsection\label{eq:genera_gener}
    \begin{aligned}
       X_i&=i\left[\left(\dfrac{\partial x'}{\partial a^i}\right)\biggm|_{a^i=0 }\dfrac{\partial }{\partial x}+\left(\dfrac{\partial y'}{\partial a^i}\right)\biggm|_{a^i=0 }\dfrac{\partial }{\partial y}+\left(\dfrac{\partial z'}{\partial a^i}\right)\biggm|_{a^i=0 }\dfrac{\partial }{\partial z}+\left(\dfrac{\partial t'}{\partial a^i}\right)\biggm|_{a^i=0 }\dfrac{\partial }{\partial t}\right]\\
       &=i\left(\dfrac{\partial x'^\mu}{\partial a^i}\right)\biggm|_{a^i=0 }\dfrac{\partial }{\partial x^\mu}
    \end{aligned}
\end{equation}

A questo punto, con la formula precedente, possiamo studiare la rappresentazione infinito dimensionale per i boost di Lorentz. Consideriamo dapprima un boost lungo x:
\begin{equation}
    \begin{cases}
         x' = \gamma(x+vt)\\
         y'=y\\
         z'=z
         \\
      t'=\gamma(t+\dfrac{v}{c^2} x)
    \end{cases}\,
\end{equation}
considerando la velocità come parametro facciamo le derivate:
\begin{equation}
    \begin{gathered}
        \dfrac{\partial x'}{\partial v}=\gamma t-\dfrac{x+vt}{(1-\frac{v^2}{c^2})^\frac{3}{2}}\left(-\dfrac{1}{2}\right)\left(2\dfrac{v}{c^2}\right)\biggm|_{v=0 }=t\\
        \dfrac{\partial t'}{\partial v}=\gamma \dfrac{x}{c^2}-\dfrac{t+\frac{v}{c^2} x}{(1-\frac{v^2}{c^2})^\frac{3}{2}}\left(-\dfrac{1}{2}\right)\left(2\dfrac{v}{c^2}\right)\biggm|_{v=0 }=\dfrac{x}{c^2}
    \end{gathered}
\end{equation}
quindi i generatori sono:
\begin{equation}
    \begin{gathered}
        K_x=-i\left(t\dfrac{\partial }{\partial x}+x\dfrac{\partial }{\partial t}\right)\\
          K_y=-i\left(t\dfrac{\partial }{\partial y}+y\dfrac{\partial }{\partial t}\right)\\
            K_z=-i\left(t\dfrac{\partial }{\partial z}+z\dfrac{\partial }{\partial t}\right)
    \end{gathered}
\end{equation}
possiamo verificare che l'algebra dei generatori è l'algebra di Lie:
\begin{equation}
    \begin{gathered}
        \left[J_i,J_j\right]=i\epsilon_{ijk}J_k\\
        \left[J_i,K_j\right]=i\epsilon_{ijk}K_k\\
         \left[K_i,K_j\right]=-i\epsilon_{ijk}J_k
    \end{gathered}
\end{equation}

Mettendo assieme quanto fatto possiamo determinare un rappresentazione infinito dimensionale per il gruppo di Lorentz.
Partiamo a monte considerando la trasformazione di Lorentz 
\begin{equation}
  x^{\mu}\xrightarrow[\text{}]{\text{T}}x'^{\mu}=\Lambda\indices{^\mu_\nu} x^{\nu}
\end{equation}
consideriamo che la trasformazione sia infinitesima, ovvero:
\begin{equation}\phantomsection\label{eq:tras_inf}
  \Lambda\indices{^\mu_\nu}=\delta{^\mu_\nu}+\mathcal{E}{^\mu_\nu} \quad\text{con} \quad |\mathcal{E}{^\mu_\nu}|<<1
\end{equation}
Ricordiamoci che la trasformazione deve conservare la metrica:
\begin{equation}
\eta_{\alpha\beta}=\Lambda^\mu_\alpha\Lambda^\nu_\beta\eta_{\mu\nu}
\end{equation}
sostituendo:
\begin{equation}
\begin{gathered}
\eta_{\alpha\beta}=(\delta^\mu_\alpha+\mathcal{E}^\mu_\alpha)(\delta^\nu_\beta+\mathcal{E}^\nu_\beta)\eta_{\mu\nu}
\\
\eta_{\alpha\beta}=\eta_{\alpha\beta}+\mathcal{E}^\nu_\beta\eta_{\alpha\nu}+\mathcal{E}^\mu_\alpha\eta_{\mu\beta}
\\
\mathcal{E}^\nu_\beta\eta_{\alpha\nu}+\mathcal{E}^\mu_\alpha\eta_{\mu\beta}=0\\
\mathcal{E}_{\beta\alpha}+\mathcal{E}_{\alpha\beta}=0 \implies \mathcal{E}_{\beta\alpha}=-\mathcal{E}_{\alpha\beta}
\end{gathered}
\end{equation}
concludiamo che è antisimmetrica e per questo dipenderà da 6 parametri indipendenti.


La dalla definizione \eqref{eq:tras_inf} abbiamo che la  trasformazione infinitesima si presenta come:
\begin{equation}
 x'^\mu=\delta{^\mu_\nu}x^\nu+\mathcal{E}{^\mu_\nu}x^\nu
\end{equation}
da cui ricaviamo:
\begin{equation}\phantomsection\label{eq:diff_tras}
 \delta x^\mu=x'^\mu-x^\mu=\mathcal{E}{^\mu_\nu}x^\nu=\mathcal{E}{^{\mu\rho}}x_\rho
\end{equation}
che possiamo riscrivere:
\begin{equation}
 \delta x^\mu=i\dfrac{1}{2}\mathcal{E}{^{\sigma\rho}}L{_{\sigma\rho}}x^\mu
\end{equation}
ove
\begin{equation}
 L{_{\sigma\rho}}\coloneqq i(x_\rho\partial_\sigma-x_\sigma\partial_\rho)
\end{equation}
Verifichiamo l'uguaglianza tra le due:
\begin{equation}
\begin{aligned}
 \delta x^\mu&=i\dfrac{1}{2}\mathcal{E}{^{\sigma\rho}}L{_{\sigma\rho}}x^\mu=i\dfrac{1}{2}\mathcal{E}{^{\sigma\rho}}i(x_\rho\partial_\sigma-x_\sigma\partial_\rho)x^\mu\\
 &=-\dfrac{1}{2}\mathcal{E}{^{\sigma\rho}}(x_\rho\delta^\mu_\sigma-x_\sigma\delta^\mu_\rho)=-\dfrac{1}{2}\mathcal{E}{^{\mu\rho}}x_\rho+\dfrac{1}{2}\mathcal{E}{^{\sigma\mu}}x_\sigma\\
 &=\dfrac{1}{2}\mathcal{E}{^{\rho\mu}}x_\rho+\dfrac{1}{2}\mathcal{E}{^{\rho\mu}}x_\rho=\mathcal{E}{^{\rho\mu}}x_\rho
\end{aligned}
\end{equation}
Considerando le componenti abbiamo:
\begin{equation}
\begin{gathered}
 L{_{0k}}= i(t\partial_k-x_k\partial_0)=i(t\partial_k+x^k\partial_0)\\
  L{_{ij}}= i(x_i\partial_j-x_j\partial_i)=-i(x^i\partial_j-x^j\partial_i)
\end{gathered}
\end{equation}
in questo modo abbiamo trovato i sei generatori precedenti, però scritti in forma covariante. L'algebra del gruppo di Lorentz in forma covariante è:
\begin{equation}
        \left[L_{\mu \nu},L_{\rho \sigma}\right]=i\eta_{\nu \rho}L_{\mu \sigma}-i\eta_{\mu \rho}L_{\nu \sigma}-i\eta_{\nu \sigma}L_{\mu \rho}+i\eta_{\mu \sigma}L_{\nu \rho}\\
\end{equation}

Approfondiamo questa algebra. Se definiamo un nuovo operatore a partire da quelli studiati in precedenza, in particolare:
\begin{equation}
    N_i=\dfrac{1}{2}(j_i+iK_i)
\end{equation}
otteniamo la seguente algebra dei $N_i$:
\begin{equation}
    \begin{gathered}
        \left[N_i,N_j^+\right]=0\\
        \left[N_i,N_j\right]=i\epsilon_{ijk}N_k\\
         \left[N_i^+,N_j^+\right]=i\epsilon_{ijk}N_k^+
    \end{gathered}
\end{equation}
tali proprietà dell'algebra determinano una, così detta, \textit{algebra diagonale}.

Quindi abbiamo trovato che, in termini di questi operatori, il gruppo di Lorentz è descritto da due insiemi di operatori commutanti tra loro, ciascuno soggetto ad un algebra di Lie. Di fatto abbiamo $SO(3,1)\simeq SU(2)\otimes SU(2)$.


\subsection{Gruppo di Poincaré}
Nella fisica delle particelle elementari abbiamo che il gruppo fondamentale è quello di Poincaré. Le trasformazioni di Poincaré sono trasformazioni di Lorentz a cui si aggiungono le traslazioni:
\begin{equation}
  x^\mu\xrightarrow[\text{}]{\text{P}}x'^{\mu}=\Lambda\indices{^\mu_\nu} x^{\nu} +a^\mu
\end{equation}
formano un gruppo. I parametri saranno 6 per Lorentz e 4 per le traslazioni, quindi 10 parametri totali; questo significa 10 generatori.

Consideriamo il caso particolare nel quale la trasformazione consiste solo di una traslazione rispetto a $x$:
\begin{equation}
  x\xrightarrow[\text{}]{\text{P}}x'=x-a
\end{equation}
Per determinare il generatore corrispondente riprendiamo la \eqref{eq:genera_gener}:
\begin{equation}
  P_x=i(-1)\dfrac{\partial}{\partial x}=-i\dfrac{\partial}{\partial x}
\end{equation}
 a meno di un $\hbar$ abbiamo trovato l'operatore impulso in meccanica quantistica. Concludiamo che l'impulso genera traslazioni. Quindi in generale varrà:
 \begin{equation}\phantomsection\label{eq:op_poincare}
  P_\mu=-i\dfrac{\partial}{\partial x^\mu}
\end{equation}
che insieme a 
\begin{equation}
 L{_{\sigma\rho}}= i(x_\rho\partial_\sigma-x_\sigma\partial_\rho)
\end{equation}
formano tutti i generatori del gruppo di Poincaré, i quali avranno l'algebra:
\begin{equation}
\begin{gathered}
        \left[L_{\mu \nu},L_{\rho \sigma}\right]=i\eta_{\nu \rho}L_{\mu \sigma}-i\eta_{\mu \rho}L_{\nu \sigma}-i\eta_{\nu \sigma}L_{\mu \rho}+i\eta_{\mu \sigma}L_{\nu \rho}\\
        \left[P_{\mu},P_\nu\right]=0\\
        \left[P_\mu,L_{\rho \sigma}\right]=i(\eta_{\mu \rho}P_\sigma-\eta_{\mu \sigma}P_\rho)
\end{gathered}
\end{equation}